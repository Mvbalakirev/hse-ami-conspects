\section{Первообразная}

\begin{defn}{}{}
    Пусть \( F, f: (a, b) \to \RR \) и \( \forall x \in (a, b) F'(x) = f(x) \)

    Тогда \( F \) называют (точной) первообразной \( f \) на \( (a, b) \)

    Если равенство \( F'(x) = f(x) \) не выполнено для конечного числа точек интервала \( (a, b) \),
    то \( F \) называют (обобщенной) первообразной \( f \) на \( (a, b) \)

    (\( F \in C(a, b) \))
\end{defn}

\begin{thrm}{}{}
    Пусть \( F_1, F_2 \) --- две обобщенные первообразные для \( f \) на \( (a, b) \),
    тогда \( \exists C \in \RR : \forall x \in (a, b) F_1(x) = F_2(x) + C \)
\end{thrm}

Доказательство:

Пусть \( c_1 c_2, \ldots, c_n \in (a, b) \) --- точки, в которых равенство
либо \( \not\exists F_1' \), либо \( \not\exists F_2' \), либо \( F_1'(c_i) \neq f(c_i) \),
либо \( F_2'(c_i) \neq f(c_i) \).

Пусть \( c_1 < c_2 < \ldots < c_n \).

Значит на \( (a, c_1), (c_1, c_2), \ldots, (c_n, b) \) \( F_1, F_2 \) --- точные первообразные.

Поэтому на этих точках \( F_1'(x) - F_2'(x)  = f(x) - f(x) = 0 \).

Поэтому на каждом интервале \( F_1(x) - F_2(x) = const \).
Поскольку разность функций тоже непрерывна, константы должны совпасть, поэтому \( F_1 = F_2 + C \)

\begin{defn}{Неопределенный интеграл}{}
    Произвольная первообразная \( f \) на на \( (a, b) \) называется неопределенным интегралом
    \( f \) на \( (a, b) \) и обозначается \( \displaystyle \int f(x) dx \)

    Если известно, что \( F \) --- точная первообразная \( f \) на \( (a, b) \),
    то \( \displaystyle \exists C \in \RR : \int f(x) dx = F(x) + C \)
\end{defn}

Fun facts:
\begin{enumerate}
    \item
        \[
            \int dF(x) = F(x) + C
        \]
    \item
        \[
            d \left( \int f(x) dx \right) = dF(x) = f(x) dx
        \]
    \item
        \[
            \left( \int f(x) dx \right)' = f(x)
        \]
\end{enumerate}
