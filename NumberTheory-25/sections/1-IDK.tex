\section{Вступление}

Сложность некоторых основных алгоритмов:
\begin{itemize}
    \item
        Логарифм

        \( q \geq 2 \)

        \( Lq(n) = \left\lfloor \log_q n \right\rfloor + 1 \)

        \( Lq(0) = 1 \)

        \( Lq(n) = O(Lq'(n)) \)
    \item
        Сложение

        \( a + b \) за \( O(\max(L(a), L(b))) \)
    \item
        Умножение

        \( M(n) = O(n^2) \) --- столбик

        \( M(n) = O(n^{\log_2 3}) \) --- алгоритм Карацубы

        \( M(n) = O_{\varepsilon} (n^{1 + \varepsilon}) \) --- алгоритм Тома-Кука

        \( M(n) = O(n \log n \log \log n) \) --- алгоритм Шенхаге-Штрассена

        \( M(n) = O(n \log n) \) (2019) --- чтобы обогнать предыдущий алгоритм, нужно число порядка \( \log n = 2^{7 \cdot 10^{38}} \)
\end{itemize}

\newpage

\section{Алгоритм Евклида}

\begin{defn}{}{}
    \( a_1, \ldots, a_n \in \ZZ \) не равные одновременно \( 0 \)

    Тогда их НОД-ом называется наибольшее число \( d \), которое делит их всех, и обозначается \( (a_1, \ldots, a_n) \)
\end{defn}

\( (a, b) = ? \)

\( a = bq + r \), \( 0 \leq r < b \)

\( (a, b) = (b, r) \)

Остается сделать так несколько раз:
\begin{gather*}
    \begin{cases}
        m_0 = a_0 m_1 + m_2
        \\
        m_1 = a_1 m_2 + m_3
        \\
        \ldots
        \\
        m_{k - 1} = a_{k - 2} m_{k - 1} + m_k
        \\
        m_{k - 1} = a_{k - 1} m_k
        \\
        m_k = d
    \end{cases}
    \quad
    m_1 > m_2 > \ldots > m_k > 0
\end{gather*}

\begin{lemma}{}{}
    Пусть \( m_0 \geq m_1 \), тогда \( k = O(\log m_1) \)
\end{lemma}

Действительно: \( m_{i - 1} = a_{i - 1} m_i + m_{i + 1} \geq m_i + m_{i + 1} \geq 2 m_{i + 1} \)

Нетрудно убедиться, что взятие модуля через деление в столбик занимает \( O(L(b) \cdot (L(q) + 1)) = O(L(b) (L(a) - L(b) + 1)) \)

\begin{thrm}{}{}
    Сложность алгоритма Евклида, примененного к числам \( a, b \) с длинами \( L(a), L(b) \leq n \) есть \( O(n^2) \)
\end{thrm}
\begin{gather*}
    L(m_1) (L(m_0) - L(m_1) + 1) + L(m_2) (L(m_1) - L(m_2) + 1) + \ldots
    \leq
    \\
    \leq
    L(m_1) (L(m_0) - L(m_1) + 1 + L(m_1) - L(m_2) + 1 + \ldots)
    \leq
    \\
    \leq
    L(m_1) (L(m_0) + k)
    =
    O(L(m_1) L(m_0))
\end{gather*}

\begin{remark}{}{}
    Существуют более быстрые варианты алгоритма Евклида

    На сегодняшний день известна оценка сложности \( O(M(n) \log n) \)

    С алгоритмом Шенхаге-Штрассена, получим \( O(n \log^2 n \log \log n) \)
\end{remark}

\section{Группы, кольца и поля}

\begin{defn}{Группа}{}
    Множество \( (G, *) \) называется группой, если выполняется \( 3 \) свойства:
    \begin{enumerate}
        \item
            \( (a * b) * c = a * (b * c) \) --- ассоциативность
        \item
            \( \exists e : a * e = e * a = a \) --- нейтральный элемент
        \item
            \( \forall a \in G \ \exists b : a * b = b * a = e \) --- обратный элемент
    \end{enumerate}
\end{defn}

\begin{example}{}{}
    \begin{itemize}
        \item
            \( G = \{ e \} \)
        \item
            \( G = \{ \ZZ, + \} \)
        \item
            \( G = \{ R^*, \cdot \} \) --- действительные числа без нуля
        \item
            \( Isom(E^2) \) --- движения плоскости (\( E^2 = \RR^2 \) --- Евклидова плоскость)
        \item
            \( S_n \) --- множество перестановок 
    \end{itemize}
\end{example}

\begin{defn}{Абелева группа}{}
    Если \( \forall a, b \in G \) верно \( a * b = b * a \), группа называется коммутативной или абелевой.
\end{defn}

\begin{defn}{Кольцо}{}
    Множество \( R \) с бинарными операциям \( + \) и \( \cdot \) называется кольцом, если:
    \begin{enumerate}
        \item
            \( (R, +) \) --- абелева группа
        \item
            \( a \cdot (b \cdot c) = (a \cdot b) \cdot c \) --- ассоциативность умножения
        \item
            \( a \cdot (b + c) = a \cdot b + a \cdot c \) и \( (b + c) a = b \cdot a + c \cdot a \) --- дистрибутивность
    \end{enumerate}
\end{defn}

\begin{example}{}{}
    \begin{itemize}
        \item
            \( R \) --- кольцо, тогда \( R[x] \) --- тоже кольцо
        \item
            \( R = \{ 0 \} \)
        \item
            \( (\ZZ, +, \cdot), (\RR, +, \cdot), (M_n(\RR), +, \cdot) \)
        \item
            \( \ZZ_m \) --- кольцо вычетов по \( \mod m \)
        \item
            \( \RR[[x]] \) --- кольцо формальных степенных рядов над \( \RR \)
    \end{itemize}
\end{example}

\begin{defn}{}{}
    \begin{enumerate}
        \item
            Если \( \exists 1 \in R : 1 \cdot a = a \cdot 1 = a \), то \( R \) называют кольцом с единицей
        \item
            Если \( \forall a, b \in R \) \( a \cdot b = b \cdot a \), то \( R \) называют коммутативным кольцом
    \end{enumerate}
\end{defn}

\begin{example}{}{}
    \( 2 \ZZ = \{ 2a : a \in \ZZ \} \) --- кольцо без \( 1 \)
\end{example}

\begin{defn}{}{}
    Если \( R \) --- кольцо с \( 1 \), то \( a \in R \) называют обратимым элементом, если \( \exists b : a \cdot b = 1 = b \cdot a \)
\end{defn}

\begin{defn}{Поле}{}
    Если в кольце \( R \) с \( 1 \) любой ненулевой элемент обратим, то \( R \) называют полем
\end{defn}

\begin{example}{Поля}{}
    \( \CC, \RR, \QQ \)
\end{example}

\begin{example}{Кольца, не являющиеся полями}{}
    \( M_n(\RR), 2\ZZ, \RR[x], \RR[[x]] \)
\end{example}

\begin{thrm}{Основная теорема арифметики}{}
    Произвольное натуральное число \( n > 1 \) единственным образом (с точностью до порядка сомножителей)
    раскладывается в произведение простых:
    \[
        n = p_1^{\alpha_1} \ldots p_s^{\alpha_s}
    \]
\end{thrm}

Существование несложно показать по индукции:
если \( n \) не простое, то \( n = ab \), где \( a, b < n \),
после чего применяем предположение индукции.

Единственность покажем от противного.
Пусть \( n \) --- наименьшее число, обладающее двумя разложениями:
\[
    n = p_1^{\alpha_1} \ldots p_s^{\alpha_s} = q_1^{\beta_1} \ldots q_t^{\beta_t}, \text{причем \( p_i \neq q_j \)}
\]

\newpage

\begin{lemma}{Лемма Евклида}{}
    \( a \: | \: bc, (a, b) = 1 \Rightarrow a \: | \: c \)
\end{lemma}

С помощью расширенного алгоритма Евклида (лемма о линейном представление НОД) получим \( au + bv = 1 \)
\begin{gather*}
    au + bv = 1
    \\
    acu + bcv = c
    \\
    a \: | \: acu, a \: | \: bcv \Rightarrow a \: | \: c
\end{gather*}

Используя лемму выше можно ``отщепляя'' \( q_j \) можно доказать, что \( p_1 | 1 \) --- противоречие. 

\begin{example}{}{}
    Не во всех кольцах число раскладывается на простыми единственным способом:
    например, в \( 2\ZZ \) верно \( 30 \cdot 2 = 60 = 6 \cdot 10 \)
\end{example}

\begin{defn}{}{}
    \( m \geq 1 \)

    Числа \( a \) и \( b \) называется сравнимыми по модулю \( m \), если \( a - b \) делится на \( m \)

    Будем обозначать  \( a \equiv b \: (\mod m) \) или \( a \equiv b \: (m) \)
\end{defn}

\begin{defn}{}{}
    Классом вычетов \( \overline{a} \) называется множество (по модулю \( m \))
    \[
        \overline{a} = \{ a + mt \: \vline \: t \in \ZZ \}
    \]
\end{defn}
