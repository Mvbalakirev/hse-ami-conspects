\section{Множества}

\[
    X = \{ a, b, c \}
\]

Принятно, что множества упорядочены, а их элементы не повторяются, то есть
\[
    \{ a, b, c \} = \{ a, b, a, c, c, b \} = \{ b, c, a \}
\]

\begin{itemize}
    \item
        \( a \in X \) --- \( a \) принадлежит \( X \)

        \( a \notin X \) --- \( a \) не принадлежит \( X \)
    \item
        \( X \subseteq Y \) --- \( X \) является подмножеством \( Y \)

        \( \forall x \ ( x \in X \Rightarrow x \in Y ) \)
    \item
        \( X = Y \Leftrightarrow ( X \subseteq Y \land Y \subseteq X ) \)
\end{itemize}

\begin{example}{}{}
    Примеры множеств:
    \begin{itemize}
        \item
            \( \NN, \ZZ, \QQ, \RR, \CC, \ldots \)
        \item
            \( \{ 1, 2, 3, 5, 7 \} \)
        \item
            \( \varnothing \) --- пустое множество
    \end{itemize}
\end{example}


\begin{example}{}{}
    Определение натуральных чисел через множества:
    \begin{gather*}
        0 = \varnothing
        \\
        n + 1 = n \cup \{ n \}
    \end{gather*}
\end{example}

\subsection{Операции с множествами}
\begin{enumerate}
    \item
        Выделение условием

        \( X \) --- множество

        \( Y = \{ x \in X \ \vline \ \phi(x) \} \)
    \item
        Объединение

        \( A, B \) --- множества

        \( A \cup B = \{ x \ \vline \ x \in A \lor x \in B \} \)
    \item
        Пересечение

        \( A, B \) --- множества

        \( A \cap B = \{ x \ \vline \ x \in A \land x \in B \} \)
    \item
        Разность множеств

        \( A, B \) --- множества

        \( A \setminus B = \{ x \ \vline \ x \in A \land x \notin B \} \)
    \item
        Симметрическая разность

        \( A, B \) --- множества

        \( A \oplus B = \{ x \ \vline \ \text{\( x \) лежит в ровно одном из \( A, B \)} \} \)
    \item
        Взятие всех подмножеств \( X \)

        \( 2^X = \{ Y \ \vline \ Y \subseteq X \} \)
    \item
        Дополнение

        \( \overline{A} = U \setminus A \)
\end{enumerate}

\subsection{Парадокс Рассела}

\( R = \{ x \ \vline \ x \notin x \} \)

Лежит ли \( R \) в \( R \)
\begin{itemize}
    \item
        \( R \in R \Rightarrow R \notin R \)
    \item
        \( R \notin R \Rightarrow R \in R \)
\end{itemize}

Получили парадокс.

Мы использовали выделение условием, но мы можем выделять только из другого множества.
Но на самом деле ``множество'' всех множеств не является множеством.

Чтобы такого не происходило, существует аксиоматика Цермело-Френкеля --- \( ZF \) или \( ZFC \) (с аксиомой выбора)

\subsection{Базовые тождества}
\begin{enumerate}
    \item
        \( A \cup B = B \cup A \)
    \item
        \( (A \cup B) \cup C = A \cup (B \cup C) \)
    \item
        \( A \cap \varnothing = \varnothing \)
    \item
        \( A \cup \varnothing = A \)
    \item
        \( A \cap B = B \cap A \)
    \item
        \( (A \cup B) \setminus C = (A \setminus C) \cup (B \setminus C) \)
\end{enumerate}
