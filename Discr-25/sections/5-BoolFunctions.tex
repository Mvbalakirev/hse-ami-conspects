\section{Булевы функции}

\subsection{Определение}

\begin{defn}{Булева функция}{}
    \( \BB = \{ 0, 1 \} \)
    
    Булева функция --- это \( f : \BB^n \to \BB \)
\end{defn}

\begin{example}{}{}
    \begin{center}
        \begin{tabular}{ | *{8}{ >{\(} c <{\)} | } }
            \hline
            x & y & x \lor y & x \land y & \lnot x & x \oplus y & x \to y & x \equiv y \\
            \hline
            0 & 0 & 0 & 0 & 1 & 0 & 1 & 1 \\
            \hline
            0 & 1 & 1 & 0 & 1 & 1 & 1 & 0 \\
            \hline
            1 & 0 & 1 & 0 & 0 & 1 & 0 & 0 \\
            \hline
            1 & 1 & 1 & 1 & 0 & 0 & 1 & 1 \\
            \hline
        \end{tabular}
    \end{center}
\end{example}

Количество булевых функций от \( n \) переменных равно \( 2^{2^n} \)

\subsection{Свойства}

\begin{exercise}{}{}
    \begin{itemize}
        \item
            \( 
                x \land ( y \land z ) = ( x \land y ) \land z,
                \
                x \land y = y \land x,
                \
                x \land 0 = 0,
                \
                x \land 1 = x
            \)
        \item
            \(
                x \lor ( y \lor z ) = ( x \lor y ) \lor z,
                \
                x \lor y = y \lor x,
                \
                x \lor 0 = x,
                \
                x \lor 1 = 1
            \)
        \item
            \(
                x \oplus ( y \oplus z ) = ( x \oplus y ) \oplus z,
                \
                x \oplus y = y \oplus x,
                \
                x \oplus 0 = x,
                \
                x \oplus 1 = \lnot x
            \)
        \item
            \(
                x \land ( y \oplus z ) = ( x \land y ) \oplus ( x \land z )
            \)
        \item
            \(
                x \land ( y \lor z ) = ( x \land y ) \lor ( x \land z ),
                \
                x \lor ( y \land z ) = ( x \lor y ) \land ( x \lor z )
            \)
        \item
            \(
                \lnot ( \lnot x ) = x 
            \)
            --- закон двойного отрицания
        \item
            \(
                x \to y = \lnot y \to \lnot x
            \)
            --- контрапозиция
        \item
            \(
                \lnot ( x \land y ) = \lnot x \lor \lnot y,
                \
                \lnot ( x \lor y ) = \lnot x \land \lnot y
            \)
            --- законы де Моргана
    \end{itemize}
\end{exercise}

\newpage

\subsection{Дизъюнктивная нормальная форма}

\( \lnot x = \overline{x} \)

Для
\(
    \alpha \in \{ 0, 1 \}
    \
    x^\alpha
    =
    \begin{cases}
        x, \text{при} \ \alpha = 1
        \\
        \overline{x}, \text{при} \ \alpha = 0
    \end{cases}
    -
    \text{литерал}
\)

\begin{defn}{Конъюнкт}{}
    \[
        x_{i_1}^{\alpha_1} \land x_{i_2}^{\alpha_2} \land \ldots \land x_{i_k}^{\alpha_k} = K
    \]
\end{defn}

\begin{defn}{Дизъюнктивная нормальная форма}{}
    ДНФ --- дизъюнкция конъюнктов
    \[
        K_1 \lor K_2 \lor \ldots \lor K_n
    \]
\end{defn}

\begin{thrm}{}{}
    Каждая булева функция \( f : \BB^n \to \BB \) может быть представлена в виде ДНФ
\end{thrm}

Доказательство:

Пусть \( f( x_1, x_2, \ldots, x_n ) \not\equiv 0 \)

Если \( f( \sigma_1, \sigma_2, \ldots, \sigma_n ) = 1 \),
возьмем конъюнкт \( K = x_1^{\sigma_1} \land x_2^{\sigma_2} \land \ldots \land x_n^{\sigma_n} \).
Очевидно, он принимает значение \( 1 \) только на наборе своих степеней.

Возьмем произведение этих конъюнктов по всем наборам \( f \), где она равна \( 1 \)

Это представление называется совершенное ДНФ (СДНФ).

\begin{defn}{Дизъюнкт}{}
    \[
        x_{i_1}^{\alpha_1} \lor x_{i_2}^{\alpha_2} \lor \ldots \lor x_{i_k}^{\alpha_k} = K
    \]
\end{defn}


\begin{defn}{Конъюктивная нормальная форма}{}
    КНФ --- конъюнкция дизъюнктов
    \[
        D_1 \lor D_2 \lor \ldots \lor D_n
    \]
\end{defn}

\begin{exercise}{}{}
    Любая булева функция имеет КНФ
\end{exercise}

\subsection{Многочлен Жегалкина}

\begin{defn}{Многочлен Жегалкина}{}
    \begin{gather*}
        \bigoplus\limits_{ I = \{ i_1, i_2, \ldots, i_k \} \subseteq \{ 1, 2, \ldots, n \} }
            a_i \land x_{i_1} \land x_{i_2} \land \ldots \land x_{i_k}
        \\
        a_i \in \{ 0, 1 \}
    \end{gather*}

    По сути каждый возможный моном либо входит в сумму, либо нет.
\end{defn}

\begin{example}{}{}
    \[
        x \lor y = x \oplus y \oplus (x \land y) = x + y + xy
    \]
\end{example}

\begin{thrm}{}{}
    Каждая булева функция \( f : \BB^n \to B \) имеет представление в виде многочлена Жегалкина (от \( n \) переменных),
    и притом единственное с точностью до перестановки мономов
\end{thrm}

Доказательство:
\[
    f(x_1, x_2, \ldots, x_n) = ( x_n \land f(x_1, \ldots, x_{n - 1}, 1) ) 
    \oplus 
    ( (x_n \oplus 1) \land f(x_1, \ldots, x_{n - 1}, 0)
\]

Теперь докажем по индукции, что всякая функция представима в виде многочлена Жегалкина:

База \( n = 1 \): очевидно

Переход \( n \to n + 1 \):
\[
    f_0(x_1, \ldots, x_{n - 1}) = f(x_1, \ldots, x_{n - 1}, 0),
    \
    f_1(x_1, \ldots, x_{n - 1}) = f(x_1, \ldots, x_{n - 1}, 1)
\]

По формуле выше:

\[
    f(x_1, \ldots, x_n) 
    = 
    (x_n \land f_1(x_1, \ldots, x_{n - 1})) 
    \oplus 
    ( (x_n \oplus 1) \land f_0(x_1, \ldots, x_{n - 1}) )
\]

По индукции \( f_0, f_1 \) представимы в виде многочлена Жегалкина, остается просто раскрыть скобки.

Покажем единственность:

Всего \( 2^{2^n} \) различных булевых функций. 
Также всего различных \( 2^{2^n} \) различных многочленов Жегалкина.
Значит каждой функции сопоставлен уникальный многочлен Жегалкина.


\subsection{Замыкание системы связок}

\begin{defn}{}{}
    \( P_2 \) --- множество всех булевых функций

    \( P_2(n) \) --- множество всех булевых функций от \( x_1, \ldots, x_n \)

    \( \mathcal{F} \subseteq P_2 \) --- систему связок
\end{defn}

\begin{example}{}{}
    \begin{itemize}
        \item
            \( \{ \lor, \land, \lnot \} \)
        \item
            \( \{ \lor, \oplus, 1 \} \)
    \end{itemize}
\end{example}

\begin{defn}{}{}
    Замыкание \( \mathcal{F} \) --- это \( [ \mathcal{F} ] \), равное следующему:
    \begin{gather*}
        \mathcal{F}_0 = \mathcal{F} \cup \{ x_1, x_2, \ldots \}
        \\
        \mathcal{F}_k \to \mathcal{F}_{k + 1} : 
            \begin{cases}
                F_k \subseteq F_{k + 1}
                \\
                g = h( f_1(y_1^{(1)}, \ldots y_{l_1}^{(1)}), \ldots, f_s(y_1^{(s)}, \ldots y_{l_s}^{(s)}) ) \in F_{k + 1}
            \end{cases}
        \\
        [ \mathcal{F} ] = \bigcup\limits_{k = 0}^{\infty} \mathcal{F}_k
    \end{gather*}
\end{defn}

\begin{example}{}{}
    \begin{itemize}
        \item
            \( [ \{ \land, \lor, \lnot \} ] = P_2 \) (ДНФ)
        \item
            \( [ \{ \land, \oplus, 1 \} ] = P_2 \) (многочлены Жегалкина)
    \end{itemize}
\end{example}

\begin{defn}{}{}
    \( \mathcal{F} \subseteq P_2 \) --- полная система связок, если \( [ \mathcal{F} ] = P_2 \)
\end{defn}

\begin{exercise}{}{}
    \begin{itemize}
        \item
            \( \mathcal{F} \subseteq [ \mathcal{F} ] \)
        \item
            \( A \subseteq B \Rightarrow [ A ] \subseteq [ B ] \)
        \item
            \( [ [ \mathcal{F} ] ] = [ \mathcal{F} ] \)
    \end{itemize}
\end{exercise}

\begin{lemma}{Достаточное условие полноты системы}{}
    Пусть \( A \subseteq P_2 \) --- полная система и \( \forall f \in A \) верно, что \( f \in [ B ] \)

    Тогда \( B \) --- полная система.
\end{lemma}

Доказательство:
\[
    A \subseteq [ B ] \Rightarrow [ A ] \subseteq [ [ B ] ] = B \Rightarrow P_2 \subseteq [ B ] \Rightarrow [ B ] = P_2
\]

\begin{coroll}{}{}
    \( \{ \land, \lnot \}, \{ \lor, \lnot \} \) --- полные системы
\end{coroll}

Доказательство:

\( B = \{ \land, \lnot \} \)

\( A = \{ \land, \lor, \lnot \}, \ [ A ] = P_2 \)

\( x \lor y = \lnot ( \lnot x \land \lnot y ) \in [ B ] \)

Для второй системы аналогично

\subsection{Замкнутые классы}

\begin{defn}{}{}
    Пусть \( F \subseteq P_2 \). 
    \( F \) называется замкнутым классом, если \( [F] = F \)
\end{defn}

\begin{example}{}{}
    \begin{itemize}
        \item
            \( F = P_2 \)
        \item
            \( F = [A], \ A \in P_2, A \neq \varnothing \)
        \item
            \( F = L \)
    \end{itemize}
\end{example}

\subsubsection{Класс L}

\begin{defn}{Класс \( L \)}{}
    \( f(x_1, \ldots, x_n) \) --- линейная, если
    \[
        f(x_1, \ldots, x_n) = a_0 \oplus ( a_1 \oplus x_1 ) \oplus \ldots \oplus ( a_n \oplus x_n )
    \]
\end{defn}

\begin{defn}{}{}
    \( L \) --- все линейные функций.
\end{defn}

\begin{remark}{}{}
    \( [ L ] = L \)
\end{remark}

\begin{lemma}{Лемма о нелинейной функции}{}
    Из любой нелинейной функции \( f(x_1, \ldots x_n), n > 1 \) подстановкой вместо переменных \( 0, x, y \) 
    можно получить \( g(x, y) \notin L \)
\end{lemma}

Доказательство:

Рассмотрим полином Жегалкина для \( f \) и выберем моном наименьшей длины, большей \( 1 \).

Не умаляя общности, это моном \( x_1 x_2 \ldots x_r, r > 1 \).

Рассмотрим \( g(x, y, \ldots, y, 0, \ldots 0 \) (последние \( n - r \) --- нули).
Нетрудно убедиться, что получим \( xy + linear \Rightarrow \) получили нелинейную функцию \( g \).

\begin{coroll}{}{}
    Пусть \( f \notin L \), тогда \( x \land y \in [ \{ 0, \lnot, f \} ] \)
\end{coroll}

Доказательство:

По лемме построим \( g(x, y) \). 
Понятно, что \( g(x, y) \in [ \{ f, 0 \} ] \).
\begin{gather*}
    g(x, y) = xy + ax + by + c, \ a, b, c \in \{ 0, 1 \}
    \\
    g(x + b, y + a) = (x + b)(y + a) + a(x + b) + b(y + a) + c = 
    \\
    = xy + by + xa + ba + ax + ab + by + ab + c = xy + ab + c
    \\
    g(x + b, y + a) =
    \left[
        \begin{gathered}
            x \land y
            \\
            \lnot (x \land y)
        \end{gathered}
    \right.
    \in [ \{ f, 0, \lnot x \} ]
    \\
    \Downarrow
    \\
    x \land y \in [ \{ f, 0, \lnot x \} ]
\end{gather*}

\subsubsection{Класс S}

\begin{defn}{}{}
    Пусть \( f(x_1, \ldots, x_n) \in P_2 \)

    Тогда двойственная булевая функция к \( f \)
    --- это \( f^* = \lnot f(\lnot x_1, \ldots, \lnot x_n) \)
\end{defn}

\begin{example}{}{}
    \begin{itemize}
        \item
            \( x^* = \lnot \lnot x = x \)
        \item
            \( (x \land y)^* = \lnot (\lnot x \land \lnot y) = x \lor y \)
        \item
            \( (f^*)^* = f \)
    \end{itemize}
\end{example}

\begin{lemma}{Принцип двойственности}{}
    \( f(x_1, \ldots, x_n) = f_0(f_1(x_1, \ldots, x_n), \ldots, f_n(x_1, \ldots, x_n)) \)

    Тогда

    \( f(x_1, \ldots, x_n) = f_0^*(f_1^*(x_1, \ldots, x_n), \ldots, f_n^*(x_1, \ldots, x_n)) \)
\end{lemma}

\begin{defn}{Класс \( S \)}{}
    \( S = \{ f \in P_2 : f = f^* \} \) --- самодвойственные функции.
\end{defn}

\begin{example}{}{}
    \begin{itemize}
        \item
            \( x = x^* \)
        \item
            \( MAJ(x, y, z) = MAJ^*(x, y, z) \)
        \item
            \( (x \oplus y)^* = \lnot ( \lnot x \oplus \lnot y ) = x \oplus y \oplus 1 = x \equiv y \)
    \end{itemize}
\end{example}

\begin{lemma}{}{}
    \[ [S] = S \]
\end{lemma}

Доказательство:
\begin{gather*}
    h(x_1, \ldots, x_n) = f_0(f_1(x_1, \ldots, x_n), \ldots, f_n(x_1, \ldots, x_n))
    =
    \\
    =
    f_0^*(f_1^*(x_1, \ldots, x_n), \ldots, f_n^*(x_1, \ldots, x_n)) = h^*(x_1, \ldots, x_n) 
\end{gather*}

\begin{lemma}{Лемма о самодвойственной функции}{}
    Пусть \( f(x_1, \ldots, x_n) \notin S \)

    Тогда подстановкой вместо переменных \( x, \lnot x \) можно получить константу.
\end{lemma}

Доказательство:
\begin{gather*}
    \exists (\alpha_1, \ldots, \alpha_n) \in \BB^n : 
        f(\alpha_1, \ldots, \alpha_n) = f(\overline{\alpha_1}, \ldots, \overline{\alpha_n}) \ (f \notin S)
    \\
    x_i 
    \to
    \begin{cases}
        x, \alpha_i = 0
        \\
        \lnot x, \alpha_i = 1
    \end{cases}
    \\
    g(0) = f(\alpha_1, \ldots, \alpha_n) = f(\overline{\alpha_1}, \ldots, \overline{\alpha_n}) = g(1)
\end{gather*}

\subsubsection{Классы \( T_0 \), \( T_1 \)}

\begin{defn}{\( T_0, T_1 \)}{}{}
    \begin{gather*}
        T_0 = \{ f \in P_2 : f(0, \ldots, 0) = 0 \}
        \\
        T_1 = \{ f \in P_2 : f(1, \ldots, 1) = 1 \}
    \end{gather*}
\end{defn}

\begin{example}{}{}
    \begin{itemize}
        \item
            \( x \land y, x \lor y \in T_0, T_1 \)
        \item
            \( 1 \in T_1, 0 \in T_0 \)
        \item
            \( x \oplus y \in T_0 \setminus T_1 \)
    \end{itemize}
\end{example}

\begin{lemma}{Лемма о функции, не сохраняющей константу}{}
    \begin{enumerate}
        \item
            \( f \notin T_0 \Rightarrow f(x, \ldots, x) \in \{ 1, \lnot x \} \)
        \item
            \( f \notin T_1 \Rightarrow f(x, \ldots, x) \in \{ 0, \lnot x \} \)
    \end{enumerate}
\end{lemma}

\subsubsection{Класс M}

\begin{defn}{}{}
    \begin{gather*}
        0 < 1
        \\
        (\alpha_1, \ldots, \alpha_n ) \leq (\beta_1, \ldots, \beta_2)
        \\
        \Updownarrow
        \\
        \forall i \ \alpha_i \leq \beta_i
    \end{gather*}
\end{defn}

\begin{defn}{Монотонная функция}{}
    \( f(x_1, \ldots, x_n) \) --- монотонная, если 
    \( 
        \forall \tilde{\alpha}, \tilde{\beta} \in \BB^n : 
        ( \tilde{\alpha} < \tilde{\beta} \Rightarrow f(\tilde{\alpha}) \leq f(\tilde{\beta})) 
    \)
\end{defn}

\begin{defn}{Класс M}{}
    \[
        M = \{ f \in P_2 : f - \ \text{монотонная} \}
    \]
\end{defn}

\begin{example}{}{}
    \begin{itemize}
        \item
            \( 1, 0 \in M \)
        \item
            \( MAJ \in M \)
        \item 
            \( x \oplus y \notin M \)
    \end{itemize}
\end{example}

\begin{exercise}{}{}
    \[ [M] = M \]
\end{exercise}

\begin{lemma}{Лемма о немонотонной функции}{}
    Пусть \( f \notin M \).

    Тогда подставляя в \( f \) вместо переменных \( 0, 1, x \) 
    можно получить \( \lnot x \).
\end{lemma}

Доказательство:
\begin{gather*}
    f \notin M \Rightarrow \exists \tilde{\alpha}, \tilde{\beta} \in \BB^n : 
        \tilde{\alpha} \leq \tilde{\beta}, f(\tilde{\alpha}) = 1, f (\tilde{\beta}) = 0
    \\
    \tilde{\alpha} = (0, \ldots, 0, 1, \ldots 1, 0, \ldots 0)
    \\
    \tilde{\beta} = (0, \ldots, 0, 1, \ldots 1, 1, \ldots 1)
    \\
    \text{различающиеся позиции есть, так как значения функции на наборах различны}
    \\
    g(x) = (0, \ldots, 0, 1, \ldots 1, x, \ldots x)
    \\
    g(0) = f(\tilde{\alpha}) = 1
    \\
    g(1) = f(\tilde{\beta}) = 0
\end{gather*}

\begin{thrm}{Критерий Поста}{}
    \( F \) --- полная 
    \( 
        \Leftrightarrow F \not\subseteq L, F \not\subseteq S, F \not\subseteq T_0, F \not\subseteq T_1, F \not\subseteq M 
    \)
\end{thrm}

Доказательство:
\begin{itemize}
    \item
        \( \Rightarrow \)

        Тривиально
    \item
        \( \Leftarrow \)

        Покажем, что \( x \land y, \lnot x \in [F] \)

        Дано: 
        \( 
            \exists 
            f_M \in F \setminus M, 
            f_L \in F \setminus L, 
            f_{T_0} \in F \setminus T_0, 
            f_{T_1} \in F \setminus T_1, 
            f_S \in F \setminus S
        \)

        По лемме о функции, не сохраняющей константу:
        \begin{gather*}
            f_{T_0} (x, \ldots x)
            =
            \left[
                \begin{gathered}
                    1 \in [F]
                    \\
                    \lnot x \in [F]
                \end{gathered}
            \right.
            \\
            f_{T_1} (x, \ldots x)
            =
            \left[
                \begin{gathered}
                    0 \in [F]
                    \\
                    \lnot x \in [F]
                \end{gathered}
            \right.
        \end{gather*}
        
        Рассмотрим два случая:
        \begin{enumerate}
            \item
                \( 0, 1 \in [F] \)

                По лемме о немонотонной функции:
                \( f_M, 0, 1, x \to \lnot x \in [F] \)
            \item
                \( \lnot x \in [F] \)

                Лемма о несамодвойстенной функции:
                \( f_S, x, \lnot x \to const \Rightarrow 0, 1 \in [F] \)
        \end{enumerate}

        Значит \( 0, 1, \lnot x \in [F] \).

        По лемме о нелинейной функции и следствию из нее:

        \( x \land y \in [ \{0, f_L, \lnot x \} ] \ \subseteq [F] \)

        Получили, что \( \lnot x, x \land y \in [F] \Rightarrow [F] = [[F]] = P_2 \)
\end{itemize}

\subsection{Предполные классы}

\begin{defn}{Предполный класс}{}
    \( F \subseteq P_2 \) --- предполный класс (в \( P_2 \)), если
    \begin{itemize}
        \item
            \( [F] = F \)
        \item
            \( F \neq P_2 \)
        \item
            \( \forall g \in P_2 \setminus F \ [F \cup \{ g \}] = P_2 \)
    \end{itemize}
\end{defn}

\begin{thrm}{}{}
    В \( P_2 \) существует лишь \( 5 \) предполных классов: \( L, S, M, T_0, T_1 \)
\end{thrm}

Доказательство:

Пусть \( F \neq L, S, M, T_0, T_1 \) --- предполный.

Тогда \( [F] = F \neq P_2 \).

По критерию Поста \( F \subseteq L \ \text{или} \ S \ \text{или} \ M \ \text{или} \ T_0 \ \text{или} \ T_1 \).
Не умаляя общности, \( F \subseteq L \). 
Тогда \( \exists g_L \in L \setminus F \Rightarrow P_2 = [F \cup \{ g_L \}] \subseteq L \) --- противоречие.

Почему эти пять классов подходят?

\begin{tabular}{ | c | c | c | c | c | c | }
    \hline
    \ & \( T_0 \) & \( T_1 \) &  \( M \) & \( S \) & \( L \) \\
    \hline
    \( T_0 \) & \ & \( 0 \) & \( x \oplus y \) & \( 0 \) & \( x \land y \) \\
    \hline
    \( T_1 \) & \( 1 \) & \ & \( x \equiv y \) & \( 1 \) & \( x \land y \) \\
    \hline
    \( M \) & \( 1 \) & \( 0 \) & \ & \( 1 \) & \( x \land y \) \\
    \hline
    \( S \) & \( \lnot x \) & \( \lnot x \) & \( \lnot x \) & \ & \( MAJ( x, y, z ) \) \\
    \hline
    \( L \) & \( 1 \) & \( 0 \) & \( x \oplus y \) & \( 1 \) & \ \\
    \hline
\end{tabular}
