\section{Отношения}

\begin{defn}{Отношение}{}
    Отношение на множествах \( A, \ B \) --- это \( R \subseteq A \times B \)

    \( a \ R \ b \Leftrightarrow (a, b) \in R \)
\end{defn}

\begin{example}{}{}
    Функция --- частный пример отношения
\end{example}

\begin{defn}{}{}
    Пусть \( R \subseteq A \times B, \ S \subseteq B \times C \) --- отношения.

    Тогда \( S \circ R \) --- это отношение на \( A, \ C \).

    \( a \ (S \circ R ) \ b \Leftrightarrow \exists b \in B : a \ R \ b \land b \ S \ c \)
\end{defn}

\begin{example}{}{}
    \( A = B = C \) --- множество людей.

    \( x \ R \ y \) --- \( x \) сын \( y \)

    \( x \ S \ y \) --- \( x \) брат \( y \)

    Тогда:
    \begin{enumerate}
        \item
            \( a \ ( S \circ R ) \ c \) \( \Rightarrow \) \( c \) дядя \( a \)

            (причем работает в обе стороны)
        \item
            \( a \ ( R \circ S ) \ c \) \( \Rightarrow \) \( c \) отец \( a \)

            (работает только в одну сторону)
        \item
            \( a \ ( R \circ R ) \ c \) \( \Rightarrow \) \( c \) дедушка \( a \)
    \end{enumerate}
\end{example}

\begin{thrm}{}{}
    Пусть \( R \subseteq A \times B, \ S \subseteq B \times C, \ T \subseteq C \times D \) --- отношения.

    Тогда \( ( T \circ S ) \circ R = T \circ ( S \circ R ) \)
\end{thrm}

Доказательство:

Левая и правая части --- отношения на \( A , D \)

Левая часть:
\begin{gather*}
    a \ (( T \circ S ) \circ R) \ d
    \\
    \exists b \in B : a \ R \ b \land b \ (T \circ S) \ d
    \\
    \exists b \in B, \ c \in C : a \ R \ b \land b \ S \ c \land c \ T \ d
\end{gather*}

Правая часть:
\begin{gather*}
    a \ ( T \circ (S \circ R)) \ d
    \\
    \exists c \in C : a \ (S \circ R) \ c \land c \ T \ d
    \\
    \exists b \in B, \ c \in C : a \ R \ b \land b \ S \ c \land c \ T \ d
\end{gather*}

\begin{defn}{Отношение эквивалентности}{}
    \( R \subseteq A \times A \)

    Отношение \( R \) на множестве \( A \) --- отношение эквивалентности,
    если оно удовлетворяет следующим условиям:
    \begin{enumerate}
        \item
            Рефлексивность

            \( \forall a \in A \ (a \ R \ a) \)
        \item
            Симметричность

            \( \forall a, b \in A \ ( a \ R \ b \Rightarrow b \ R \ a) \)
        \item
            Транзитивность

            \( \forall a, b, c \in A \ ( a \ R \ b \land b \ R \ c \Rightarrow a \ R \ c ) \)
    \end{enumerate}
\end{defn}

\begin{example}{}{}
    \begin{itemize}
        \item
            \( A \) --- множество людей, \( x \ R \ y \Leftrightarrow \) \( x \) и \( y \) имеют одинаковые имена
        \item
            \( A = \ZZ, \ x \ R \ y \Leftrightarrow x \equiv_n y \)
        \item
            \( A = \NN^2, \ (x, y) \ R \ (p, q) \Leftrightarrow xq = yp \)
    \end{itemize}
\end{example}

\begin{example}{}{}
    \( A = \bigsqcup\limits_{i \in I} A_i \).

    Есть \( R \) на \( A \).

    \( x \ R \ y \Leftrightarrow \exists i \in I : x, y \in A_i \).

    Тогда \( R \) --- отношение эквивалентности.
\end{example}

\newpage

\begin{thrm}{}{}
    Пусть \( A \neq \varnothing \) и \( R \) --- отношение эквивалентности.

    Тогда существует разбиение \( A = \bigsqcup\limits_{i \in I} A_i \) такое, что
    \[
        \forall x, y \in A \ x \ R \ y \Leftrightarrow \exists i \in I : x, y \in A_i
    \]

    \( A_i \) называют классами эквивалентностями.
\end{thrm}

Доказательство:

\( a \in A \), определим \( [a] = \{ x \in A \ \vline \ a \ R \ x \} \)

Тогда
\begin{enumerate}
    \item
        \( \forall a \in A \ a \in [a] \Rightarrow A = \bigcup\limits_{a \in A} [a] \)
    \item
        \(
        \forall a, b \in A \ [a] \cap [b] =
        \begin{cases}
            \varnothing
            \\
            [a] = [b]
        \end{cases}
        \)

        Докажем это.

        Пусть \( x \in [a] \cap [b] \)

        \( a \ R \ x \land x \ R \ b \Rightarrow a \ R \ b \Rightarrow b \in [a] \).

        Но тогда \( \forall y \in [b] \ a \ R \ b \land b \ R \ y \Rightarrow a \ R \ y \Rightarrow y \in [a] \)

        Значит \( [b] \subseteq [a] \).

        Аналогично \( [a] \subseteq [b] \), поэтому \( [a] = [b] \)
    \item
        Если выкинем повторы, то получим \( A = \bigcup\limits_{i \in I} [a_i] \)
    \item
        Покажем, что \( x \ R \ y \Leftrightarrow \exists i \in I : x, y \in [a_i] \)

        \begin{itemize}
            \item
                \( \Rightarrow \)

                \( x \ R \ y \Rightarrow x \in [x], \ y \in [x] \)
            \item
                \( \Leftarrow \)

                \( x, y \in [a_i] \Rightarrow x \ R \ a_i \land a_i \ R \ y \Rightarrow x \ R \ y \)
        \end{itemize}
\end{enumerate}

\begin{remark}{}{}
    \( \{ A_i \ \vline \ i \in I \} = A \slash R \) --- фактор множество \( A \) по \( R \)
\end{remark}
