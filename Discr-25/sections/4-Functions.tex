\section{Функции}

\begin{defn}{Упорядоченная пара}{}
    \( (a, b) = \{ \{ a \}, \{ a, b \} \} \)
\end{defn}

\begin{defn}{Декартово произведение}{cartesian-product}
Декартово произведение множеств \( A \) и \( B \) обозначается \( A \times B \)
и является множеством всех упорядоченных пар \( (a, b) \), где \( a \in A, b \in B \).

\[
    A \times B = \{ (a, b) \ \vline \ a \in A, b \in B \}
\]
\end{defn}

\begin{defn}{Функция}{function}
Подмножество \( f \) декартова произведения \( X \times Y \)
называется функцией из \( X \) в \( Y \) (\(f : X \rightarrow Y \)), если
\( \forall x \in X \: \exists ! \ (x, y) \in f \)
\end{defn}

\( y = f(x) \).

\( x \) --- прообраз \( y \), \( y \) --- образ \( x \).

\( X \) --- область определения, \( Y \) --- область значений.

Когда пишут \( f : X \rightarrow Y \) обычно считают,
что \( X \) --- полная область определения, то есть
\( f \) определена на всем \( X \).
Тогда \( f \) --- тотальная функция.

\begin{defn}{Инъекция}{}
    \( f \) --- инъекция (вложение) множества \( X \) на множество \( Y \),
    если
    \[
        f(x_1) = f(x_2) \Rightarrow x_1 = x_2
    \]
\end{defn}
\begin{defn}{Сюръекция}{}
    \( f \) --- сюръекция (накрытие) множества \( X \) на множество \( Y \),
    если
    \[
        \forall y \in Y \: \exists x \in X : f(x) = y
    \]
\end{defn}
\begin{defn}{Биекция}{}
    \( f \) --- биекция множества \( X \) на множество \( Y \), если она является и сюръекцией, и инъекцией.
\end{defn}

\subsection{Композиция функций}

\begin{defn}{Композиция}{}
    Пусть есть тотальные \( f : A \to B, g : B \to C \)

    Определим функцию \( g \circ f : A \to C \)

    \( (g \circ f)(a) = g(f(a)) \)
\end{defn}

\begin{thrm}{Ассоциативность композиции}{}
    Пусть есть тотальные \( f : A \to B, g : B \to C, h : C \to D \)

    \( h \circ ( g \circ f ) = ( h \circ g ) \circ f \)
\end{thrm}

Доказательство:
\begin{gather*}
    h \circ ( g \circ f ) : A \to D
    \\
    ( h \circ ( g \circ f ) ) (a) = h ( ( g \circ f )(a) ) = h(g(f(a)))
    \\
    ( h \circ g ) \circ f : A \to D
    \\
    ( ( h \circ g ) \circ f ) (a) = (h \circ g) ( f(a) ) = h(g(f(a)))
\end{gather*}

\subsection{Критерий биективности}

\begin{defn}{Тождественная на \( X \) функция}{}
    Определим на множестве \( X \) функцию \( id_X : X \to X \)
    \[
        \forall x \in X \ id_X(x) = x
    \]

    \begin{remark}{}{}
        \( f \circ id_A = f \)

        \( id_B \circ f = f \)
    \end{remark}
\end{defn}

\begin{thrm}{Критерий биективности функций}{}
    Пусть \( f : A \to B \).

    Тогда \( f \) --- биекция \( \Leftrightarrow \exists g : B \to A : f \circ g = id_B, g \circ f = id_A \)
\end{thrm}

Доказательство:
\begin{itemize}
    \item
        \( \Rightarrow \)

        Построим \( g : B \to A \)

        \( g(b) \in f^{-1}({b}) \)

        Тогда \( f(g(b)) = b \Rightarrow f \circ g = id_B \).

        \( g(f(a)) \in f^{-1}({f(a)}) = f{a} \Rightarrow g(f(a)) = a \Rightarrow g \circ f = id_A \)
    \item
        \( \Leftarrow \)

        \( \exists g : B \to A : f \circ g = id_A, g \circ f = id_B \)

        \begin{enumerate}
            \item
                Покажем, что \( f \) --- инъекция

                \( f(a_1) = f(a_2) \Rightarrow g(f(a_1)) = g(f(a_2)) \Rightarrow id_A(a_1) = id_A(a_2) \Rightarrow a_1 = a_2 \)
            \item
                Покажем, что \( f \) --- сюръекция

                Рассмотрим \( b \in B \). Найдем \( a \in A : f(a) = b \).

                \begin{gather*}
                    ( f \circ g ) (b) = b
                    \\
                    f(g(b)) = b
                    \\
                    a = g(b) \in A - \text{подходит}
                \end{gather*}
        \end{enumerate}

        \( f \) --- сюръекция и инъекция, а значит \( f \) --- биекция.
\end{itemize}

\begin{defn}{Обратная функция}{}
    Пусть \( f : A \to B \) --- биекция

    Тогда \( g : B \to A \), для которой \( f \circ g = id_B \) и \( g \circ f = id_A \),
    называют обратной к \( f \) и обозначают \( f^{-1} \)
\end{defn}

\begin{exercise}{}{}
    Если \( f \) --- биекция, то \( f^{-1} \) единственна
\end{exercise}

\begin{lemma}{}{}
    Пусть \( f : A \to B, g : B \to C \) --- биекции

    Тогда
    \begin{itemize}
        \item
            \( g \circ f \) биективна
        \item
            \( ( g \circ f )^{-1} = f^{-1} \circ g^{-1} \)
    \end{itemize}
\end{lemma}

Доказательство:

Просто проверим критерий биективности, используя две эти функции.

\begin{exercise}{}{}
    \( f : A \to B, g : B \to C \)
    \begin{enumerate}
        \item
            если \( f, g \) --- инъекции, то \( g \circ f \) --- инъекция
        \item
            если \( f, g \) --- сюръекции, то \( g \circ f \) --- сюръекция
    \end{enumerate}
\end{exercise}

