\section{Математическая индукция}

\subsection{Что это такое?}

\begin{example}{}{}
    Докажите, что если на плоскости проведены \( n \) прямых,
    то можно раскрасить полученные области в два цвета так,
    что никакие две соседних не покрашены в один цвет.
\end{example}

Выкинем прямую, покрасим для \( n - 1 \), потом добавим прямую обратно и инвертируем полуплоскость.

Что необходимо для индукции?
\begin{enumerate}
    \item База: \( A(1) \) --- истинно
    \item Переход: \( \forall n (A(n) \Rightarrow A(n + 1)) \)
\end{enumerate}

По принципу математической индукции \( \forall n \ A(n) \) --- истинно

\begin{example}{}{}
    \( 1 + 2 + \ldots + n = \frac{n(n + 1)}{2} \)
\end{example}

База:
\[
    1 = \frac{1 \cdot 2}{2}
\]

Переход:
\[
    \frac{n(n + 1)}{2} + (n + 1) = (n + 1)(\frac{n}{2} + 1) = \frac{(n + 1)(n + 2)}{2}
\]

\begin{example}{}{}
    \( 1 + \frac{1}{2^2} + \ldots + \frac{1}{100^2} < 2 \)
\end{example}

\( A(n) = 1 + \frac{1}{2^2} + \ldots \frac{1}{n^2} \leq 2 - \frac{1}{n} \)

База:
\[
    1 \leq 1
\]

Переход:
\begin{gather*}
    1 + \frac{1}{2^2} + \ldots + \frac{1}{n^2} + \frac{1}{(n + 1)^2} \leq 2 - \frac{1}{n} + \frac{1}{(n + 1)^2} \leq \\
    \leq 2 - \frac{1}{n} + \frac{1}{n(n + 1)} = 2 - \frac{1}{n + 1}
\end{gather*}

Что нужно делать аккуратно:
\begin{itemize}
    \item Не забыть проверить базу
    \item Убедиться, что переход работает для всех \( n \) (не забыть всякие крайние случаи \( 1 \rightarrow 2 \) и т.п.)
    \item Не надо добавлять объект к конструкции на \( n \) элементах, надо выкинуть один из \( n + 1 \)
\end{itemize}

\subsection{Полная индукция}

Есть \( A(1), A(2), \ldots , A(n), \ldots \)

Главное отличие в переходе: \( (\forall k < n \ A(k)) \Rightarrow A(n) \)

\begin{example}{}{}
    Докажите, что выпуклый многоугольник можно триангулировать, причем всего будет \( n - 2 \) треугольника.
\end{example}

\subsection{Эквивалентность различных принципов индукции}

\begin{enumerate}
    \item {
        Принцип математической индукции:
        \[
        \begin{rcases}
            A(1) \\
            \forall n \ A(n) \Rightarrow A(n + 1)
        \end{rcases}
        \Rightarrow \forall n \ A(n)
        \]
    }
    \item {
        Полная индукция:
        \[
            ( ( \forall k < n \ A(k) ) \Rightarrow A(n) ) \Rightarrow \forall n \ A(n)
        \]
    }
    \item {
        Принцип наименьшего числа \( \forall S \subseteq \NN, S \neq \varnothing \)
        Тогда в \( S \). есть наименьший элемент: \( \exists s \in S : \forall t \in S \ s \leq t \)
    }
\end{enumerate}

\begin{thrm}{}{induction-equiv}
    Три вышеуказанных принципа индукции эквивалентны.
\end{thrm}

\begin{itemize}
    \item
        \( 3 \rightarrow 1 \)

        Имеем \( A(1), A(2), \ldots, A(n), \ldots \) и \( A(n) \Rightarrow A(n + 1) \)

        \( S = \{ n \in N \ \vline \ A(n) - \text{ложно} \} \)

        Либо \( S \) пустое, либо в нем есть минимальный элемент \( s \).
        Но тогда либо \( s = 1 \) и неверна база, либо неверен переход \( A(s - 1) \Rightarrow A(s) \).
        Значит \( S \) пустое и все \( A(n) \) верны.
    \item
        \( 2 \rightarrow 3 \)

        Есть \( S \neq \varnothing \). Пусть \( S \) не имеет наименьшего элемента.

        \( A(n) = n \notin S \)

        \( \forall k < n \ A(k) \Rightarrow A(n) \)

        \( \forall k < n \ k \notin S \Rightarrow n \notin S \) --- иначе в \( S \) есть минимальный элемент.

        Тогда \( \forall n \ n \notin S \Rightarrow S = \varnothing \)
    \item
        \( 1 \rightarrow 2 \)

        \( B(n) = A(1) \land A(2) \land \ldots \land A(n) \)

        \[
            \begin{rcases}
                B(1) = A(1)
                \\
                B(n) \Rightarrow A(n) \Rightarrow A(n + 1) \Rightarrow B(n + 1)
            \end{rcases}
            \Rightarrow
            \forall n \ B(n)
        \]
\end{itemize}
